% -*- root: ../assignment.tex -*-

\subsection*{Vectors}
\begin{enumerate}
    \item Is this set of vectors $\left\{\begin{bmatrix}2 \\ 1 \\ 1\end{bmatrix}, \begin{bmatrix}0 \\ 0\\ 0\end{bmatrix}, \begin{bmatrix}0 \\ 1 \\ 0\end{bmatrix}\right\}$ independent? Explain your answer.

    \item Consider a set of finite duration discrete-time real signals $X_N = \lc x\dt{n} \big\vert x\dt{n} \in \mb{R}, \,\forall 0 \leq n \leq N-1\rc$. Does this set form a vector space? Explain your answer. Would $X_N$ still be a vector spaces if the signals were binary signals? i.e. $x\dt{n} \in \mb{B}$, where $\mb{B} = \lc 0, 1\rc$ with the binary addition and multiplication operations defined as the following,

    \begin{minipage}[h]{.45\textwidth}
        \centering
        \begin{tabular}{|c|c|c|c|}
        \hline
        $a$ & $b$ & $a+b$ & $a \times b$ \\ \hline
        0 & 0 & 0   & 0     \\ \hline
        0 & 1 & 1   & 0     \\ \hline
        1 & 0 & 1   & 0     \\ \hline
        1 & 1 & 0   & 1     \\ \hline
        \end{tabular}\\
        \captionof{table}{\small{Addition and Multiplication operation for binary numbers.}}
    \end{minipage}

    \item Prove the following for $\mathbf{x}, \mathbf{y} \in \mathbb{R}^n$
    \begin{enumerate}
        \item {\small \textbf{Triangle Inequality}}:
        $$\left\lVert \mathbf{x} + \mathbf{y}\right\rVert \leq \left\lVert \mathbf{x}\right\rVert + \left\lVert \mathbf{y}\right\rVert$$
        \item {\small \textbf{Backward Triangle Inequality}}:
        $$\left\lVert \mathbf{x} - \mathbf{y}\right\rVert  \geq \left\lvert \left\lVert \mathbf{x}\right\rVert - \left\lVert \mathbf{y}\right\rVert \right\rvert$$
        \item {\small \textbf{Parallelogram Identity}}:
        $$\frac{1}{2} \left(\left\lVert \mathbf{x} + \mathbf{y}\right\rVert^2 + \left\lVert \mathbf{x} - \mathbf{y}\right\rVert^2 \right) = \left\lVert \mathbf{x}\right\rVert^2 + \left\lVert \mathbf{y}\right\rVert^2$$
    \end{enumerate}

    \item Consider a set of vectors $\mathbf{x}, \mathbf{y} \in \mathbb{R}^n$. When is $\left\lVert \mathbf{x} - \mathbf{y} \right\rVert = \left\lVert \mathbf{x} + \mathbf{y}\right\rVert$? What can you say about the geometry of the vectors $\mathbf{x},\,\mathbf{y},\,\mathbf{x} - \mathbf{y}$ and $\mathbf{x} + \mathbf{y}$?
    
    \item If $S_1, S_2 \subseteq V$ are subspaces of $V$, the is $S_1 \cap S_2$ a subspace? Demonstrate your answer.

    \item Consider two sets of vectors $V = \lc \mf{v}_1, \mf{v}_2, \ldots \mf{v}_n\rc$ and $W = \lc \mf{w}_1, \mf{w}_2, \ldots, \mf{w}_n, \mf{u}\rc$. Prove that if $span\ct{V} = span\ct{W}$, then $\mf{u} \in span\ct{V}$.
    
    \item Prove that the sum of two subspaces $S_1, S_2 \subseteq V$ is a subspace.
    
    \item Consider a vector $\mathbf{v} = \begin{bmatrix*}v_1\\v_2\\\vdots\\v_n\end{bmatrix*}$. Express the following in-terms of inner product between a vector $\mathbf{u}$ and $\mathbf{v}$, and in each case specify the vector $\mathbf{u}$.
    \begin{enumerate}
        \item $\sum_{i=1}^nv_i$
        \item $\frac{1}{n}\sum_{i=1}^nv_i$
        \item $\sum_{i=1}^nv_ix^{\left(n - i\right)}$, where $x \in \mathbb{R}$
        \item $\frac{1}{n-1}\sum_{i=1}^n\left(v_i - \frac{1}{n}\sum_{i=1}^nv_i \right)^2$
        \item $\frac{1}{5}\sum_{i=3}^5 v_i$
        \item $\sum_{i=1}^{n-1} \left(v_{i+1} - v_i\right)$
        \item $\sqrt{v_n}$
        \item $\sum_{i=1}^nw_iv_i^2$
    \end{enumerate}
    
    \item Which of the following are linear functions of $\left\{x_1, x_2, \ldots,x_n\right\}$?
    \begin{enumerate}
        \item $\min_i \left\{x_i\right\}_{i=1}^{n}$
        \item $\left(\sum_{i=1}^n x_i^2\right)^{1/2}$
        \item $x_6$
    \end{enumerate}
    
    \item Consider a linear function $f: \mathbb{R}^n \rightarrow \mathbb{R}$. Prove that every linear function of this form can be represented in the following form.
    \[ y = f\left(\mathbf{x}\right) = \mathbf{w}^T\mathbf{x} = \sum_{i=1}^{n}w_ix_i, \,\,\,\,\, \mathbf{x}, \mathbf{w} \in \mathbb{R}^n \]

    \item An \textit{affine} function $f$ is defined as the sum of a linear function and a constant. It can in general be represented in the form, 
    \[ y = f\left(\mathbf{x}\right) = \mathbf{w}^T\mathbf{x} + \beta, \,\,\,\,\, \mathbf{x}, \mathbf{w} \in \mathbb{R}^n, \, \beta \in \mathbb{R} \]
    Prove that affine functions are not linear. Prove that any affine function can be represented in the form $\mathbf{w}^T\mathbf{x} + \beta$.

    \item Consider a function $f: \mathbb{R}^3 \rightarrow \mathbb{R}$, such that,
    \[ f\left(\begin{bmatrix*}[r]1\\0\\0\end{bmatrix*}\right) = 2; \,\,
    f\left(\begin{bmatrix*}[r]0\\1\\0\end{bmatrix*}\right) = -3; \,\,
    f\left(\begin{bmatrix*}[r]0\\0\\1\end{bmatrix*}\right) = 1; \,\,\]
    Can you determine the following values of $f\left(\mathbf{x}\right)$, if you are told that $f$ is linear?
    \[ f\left(\begin{bmatrix*}[r]2\\2\\-2\end{bmatrix*}\right) = ?; \,\,\,
     f\left(\begin{bmatrix*}[r]-1\\2\\0\end{bmatrix*}\right) = ?; \,\,\,
     f\left(\begin{bmatrix*}[r]0.5\\0.6\\-0.1\end{bmatrix*}\right) = ?; \,\,\,\]
     Can you find out these values if you are told that $f$ is affine?
    
    \item For the previous question, (a) assume that $f$ is linear and find out $w \in \mathbb{R}^3$, such that $f\left(\mathbf{x}\right) = \mathbf{w}^T\mathbf{x}$; and (b) assume $f$ is affine and find out $\mathbf{w}, \beta$ such that $f\left(\right) = \mathbf{w}^T\mathbf{x} + \beta$.

    \item Consider the weighted norm of vector $\mathbf{v}$, defined as,
    \[ \left\lVert \mathbf{v}\right\rVert_{\mathbf{w}}^2 = \sum_{i=1}^{n}w_iv_i^2; \,\,\,\,\, \mathbf{w}=\begin{bmatrix*}w_1\\w_2\\\vdots\\w_n\end{bmatrix*} \]
    Is this a valid norm?\\

    \item Prove that the following modified version of the Cauchy-Bunyakovski-Schwartx Inequality is true.
    \[ \left\lvert \sum_{i=1}^{n} u_iv_iw_i \right\rvert \leq \left\lVert \mathbf{u}\right\rVert_\mathbf{w} \left\lVert \mathbf{v}\right\rVert_\mathbf{w} \]

    \item Consider a basis $B = \left\{\mathbf{b}_i\right\}_{i=1}^{n}$ of $R^n$. Let the vector $\mathbf{x}$ with the following representations in the standard and $B$ basis.
    \[ \mathbf{x} = \begin{bmatrix*}x_1\\x_2\\\vdots\\x_n\end{bmatrix*} = \sum_{i=1}^{n}x_i\mathbf{e}_i \,\,\,\,\, \text{and} \,\,\,\,\, \mathbf{x}_b =  \begin{bmatrix*}x_{b1}\\x_{b2}\\\vdots\\x_{bn}\end{bmatrix*} = \sum_{i=1}^{n}x_{bi}\mathbf{b}_i \]

    Evaluate the $\left\lVert \mathbf{x}\right\rVert_2^2$ and $\left\lVert \mathbf{x}_b\right\rVert_2^2$. Determined what happens to $\left\lVert \mathbf{x}_b\right\rVert_2^2$ under the following conditions on the basis vectors:
    \begin{enumerate}
        \item $\left\lVert \mathbf{b}_i\right\rVert = 1, \forall i$
        \item $\left\lVert \mathbf{b}_i^T\mathbf{b}_j\right\rVert = \begin{cases}
        1 & i = j\\
        0 & i \neq j
        \end{cases}$
    \end{enumerate}

    \item Consider a set of measurements made from adult male subjects, where their height, weight and BMI (body moass index) were recorded as stored as vectors of length three; the first element is the height in $cm$, second is the weight in $Kg$, and the alst the the BMI. Consider the following four subjects,
    \[ \mathbf{s}_1 =  \begin{bmatrix*}[r]167\\102\\36.6\end{bmatrix*}; \,\,
    \mathbf{s}_2 =  \begin{bmatrix*}[r]180\\87\\26.9\end{bmatrix*} \]
    \[ \mathbf{s}_3 =  \begin{bmatrix*}[r]177\\78\\24.9\end{bmatrix*}; \,\,
    \mathbf{s}_4 =  \begin{bmatrix*}[r]152\\76\\32.9\end{bmatrix*} \]

    You can use the distance between these vectors $\left\lVert \mathbf{s}_i - \mathbf{s}_j\right\rVert_2$ as a measure of the similiarity between the the four subjects. Generate a $4 \times 4$ table comparing the distance of each subject with respect to another subject; the diagonal elements of this table will be zero, and it will be symmetric about the main diagonal. 

    (a) Based on this table, how do the different subjects compare to each other?

    (b) How do the similarities change if the height had been measured in $m$ instead of $cm$? Can you explain this difference?

    (c) Is there a way to fix this problem? Consider the weighted norm presented in one of the earlier problems.
    \[ \left\lVert x \right\rVert_\mathbf{w} = \left(w_1x_1^2 + w_2x_2^2 + \ldots + w_nx_n^2\right)^{\frac{1}{2}} \]

    (d) What would be a good choice for $\mathbf{w}$ to address the problems with comparing distance between vectors due to change in units?

    (e) Can the angle between two vectors be used as a measure of similarity between vectors? Does this suffer from the problem of $\left\lVert \mathbf{x} \right\rVert_2$?
\end{enumerate}
% \vfill