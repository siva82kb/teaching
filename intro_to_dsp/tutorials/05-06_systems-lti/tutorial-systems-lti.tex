\documentclass[12pt]{article}

\usepackage[utf8]{inputenc}
\usepackage{geometry}
\geometry{
    a4paper,
    total={170mm,257mm},
    left=15mm,
    right=15mm,
    top=20mm,
    bottom=20mm,
}
\usepackage{multicol}
\usepackage[font=small,labelfont=bf]{caption}
\setlength{\columnsep}{0.25cm}
\usepackage[inline]{enumitem}
\usepackage{amssymb}
\usepackage{xcolor}
\usepackage{mathtools} 
\setlength{\parindent}{0em}
\setlength{\parsep}{0em}
\usepackage{tikz}
\setlength{\parskip}{0em}
\usetikzlibrary{decorations.pathmorphing,patterns}
\usepackage[american,cuteinductors]{circuitikz}
\usetikzlibrary{shapes,arrows,circuits,calc,babel}
% Definition of blocks:
\tikzset{%
  block/.style    = {draw, thick, rectangle, minimum height = 3em,
    minimum width = 3em},
  sum/.style      = {draw, circle, node distance = 2cm}, % Adder
  input/.style    = {coordinate}, % Input
  output/.style   = {coordinate} % Output
}
% Defining string as labels of certain blocks.
\newcommand{\suma}{\Large$+$}
\newcommand{\inte}{$\displaystyle \int$}
\newcommand{\derv}{\huge$\frac{d}{dt}$}

\def\mf{\ensuremath\mathbf}
\def\mb{\ensuremath\mathbb}
\def\mc{\ensuremath\mathcal}
\def\lp{\ensuremath\left(}
\def\rp{\ensuremath\right)}
\def\lv{\ensuremath\left\lvert}
\def\rv{\ensuremath\right\rvert}
\def\lV{\ensuremath\left\lVert}
\def\rV{\ensuremath\right\rVert}
\def\lc{\ensuremath\left\{}
\def\rc{\ensuremath\right\}}
\def\ls{\ensuremath\left[}
\def\rs{\ensuremath\right]}
\def\bmx{\ensuremath\begin{bmatrix*}[r]}
\def\emx{\ensuremath\end{bmatrix*}}
\def\bmxc{\ensuremath\begin{bmatrix*}[c]}
\def\emxc{\ensuremath\end{bmatrix*}}
% \def\t{\lp t\rp}
% \def\k{\ls k\rs}

\newcommand{\demoex}[2]{\onslide<#1->\begin{color}{black!60} #2 \end{color}}
\newcommand{\demoexc}[3]{\onslide<#1->\begin{color}{#2} #3 \end{color}}
\newcommand{\anim}[3]{\onslide<#1->{\begin{color}{#2!60} #3 \end{color}}}
\newcommand{\ct}[1]{\lp #1\rp}
\newcommand{\dt}[1]{\ls #1\rs}

\renewcommand{\familydefault}{\sfdefault}

\begin{document}
\begin{center}
    \begin{Large}
        \textbf{Introduction to DSP: Systems \& LTI Systems - Tutorial}
    \end{Large}
\end{center}
\vspace{0.2cm}

\begin{enumerate}
    \item Consider the following discrte-time signal,
    \[ x[n] = \begin{cases}
    -2, & n = -2\\
     0, & n = -1\\
     1, & n = 0\\
     3, & n = 1\\
    -1, & n = 2\\
     1, & n = 0\\
     0, & \text{Otherwise}
     \end{cases} \]
    Compute the followint signals.
    \begin{enumerate}
        \item $x\ls -n \rs = 1$ \vspace{3cm}
        \item $x\ls n + 3 \rs = 1$ \vspace{3cm}
        \item $x\ls -n + 1\rs$ \vspace{3cm}
        \item $x\ls -n -2 \rs$ \vspace{3cm}
    \end{enumerate}
    \newpage

    \item Find if the following systems satisfy the properties of linearity, time-invariance, causality, and stability. Compute the impulse response of the systems that are linear and time-invariant.
    \begin{enumerate}
        \item $y\ls n \rs = \sum_{k=0}^{\infty} x[n-k]$ \vspace{12cm}
        \item $y\ls n \rs = \sum_{k=-3}^{2} x[n+k] \cdot x[n-k]$ \vspace{12cm}
        \item $y\ls n\rs = y[n-1] + 0.1 \cdot x[n]$ \vspace{12cm}
        \item $y\ls n \rs = n \cdot x[n] + (n-1) \cdot x[n-1]$
    \end{enumerate}
    \newpage

    \item Compute of the output an LTI system with the following impulse response
    \[ h[n] = \begin{cases} 0, & n < 0 \\ 3, & n = 0 \\ 2, & n = 1 \\ 1, & n = 2 \\ 0, & n > 2\end{cases}\] 
    \begin{enumerate}
        \item $x[n] = \delta[n] + \delta[n - 3]$ \newpage
        \item $x[n] = u[n]$ \newpage
        \item $x[n] = \sin\lp 0.5\pi n\rp u[n]$ \newpage
        \item $x[n] = 1, \forall n$ \newpage
        \item $x[n] = \lp 0.5 \rp^n, \forall n$ \newpage
        \item $x[n] = \lp 0.5 \rp^nu[n], \forall n$ \newpage
    \end{enumerate}
\end{enumerate}

\end{document}