\documentclass[12pt]{article}

\usepackage[utf8]{inputenc}
\usepackage{geometry}
\geometry{
    a4paper,
    total={170mm,257mm},
    left=15mm,
    right=15mm,
    top=20mm,
    bottom=20mm,
}
\usepackage{multicol}
\usepackage[font=small,labelfont=bf]{caption}
\setlength{\columnsep}{0.25cm}
\usepackage[inline]{enumitem}
\usepackage{amssymb}
\usepackage{xcolor}
\usepackage{mathtools} 
\setlength{\parindent}{0em}
\setlength{\parsep}{0em}
\usepackage{tikz}
\setlength{\parskip}{0em}
\usetikzlibrary{decorations.pathmorphing,patterns}
\usepackage[american,cuteinductors]{circuitikz}
\usetikzlibrary{shapes,arrows,circuits,calc,babel}
% Definition of blocks:
\tikzset{%
  block/.style    = {draw, thick, rectangle, minimum height = 3em,
    minimum width = 3em},
  sum/.style      = {draw, circle, node distance = 2cm}, % Adder
  input/.style    = {coordinate}, % Input
  output/.style   = {coordinate} % Output
}
% Defining string as labels of certain blocks.
\newcommand{\suma}{\Large$+$}
\newcommand{\inte}{$\displaystyle \int$}
\newcommand{\derv}{\huge$\frac{d}{dt}$}

\def\mf{\ensuremath\mathbf}
\def\mb{\ensuremath\mathbb}
\def\mc{\ensuremath\mathcal}
\def\lp{\ensuremath\left(}
\def\rp{\ensuremath\right)}
\def\lv{\ensuremath\left\lvert}
\def\rv{\ensuremath\right\rvert}
\def\lV{\ensuremath\left\lVert}
\def\rV{\ensuremath\right\rVert}
\def\lc{\ensuremath\left\{}
\def\rc{\ensuremath\right\}}
\def\ls{\ensuremath\left[}
\def\rs{\ensuremath\right]}
\def\bmx{\ensuremath\begin{bmatrix*}[r]}
\def\emx{\ensuremath\end{bmatrix*}}
\def\bmxc{\ensuremath\begin{bmatrix*}[c]}
\def\emxc{\ensuremath\end{bmatrix*}}
% \def\t{\lp t\rp}
% \def\k{\ls k\rs}

\newcommand{\demoex}[2]{\onslide<#1->\begin{color}{black!60} #2 \end{color}}
\newcommand{\demoexc}[3]{\onslide<#1->\begin{color}{#2} #3 \end{color}}
\newcommand{\anim}[3]{\onslide<#1->{\begin{color}{#2!60} #3 \end{color}}}
\newcommand{\ct}[1]{\lp #1\rp}
\newcommand{\dt}[1]{\ls #1\rs}

\renewcommand{\familydefault}{\sfdefault}

\begin{document}
\begin{center}
    \begin{Large}
        \textbf{Introduction to DSP: Geometric Signal Theory \& Sampling Theorem - Tutorial}
    \end{Large}
\end{center}
\vspace{0.2cm}

\begin{enumerate}
    \item Which of the following signal pairs are orthogonal? We assume that all signals here are of finite length $N$, i.e. $0 \leq n < N$.
    \begin{enumerate}
        \item $x\ls n \rs = 1$ and $y\ls n \rs = \lp -1 \rp^n$. \vspace{3cm}
        \item $x\ls n \rs = 1$ and $y\ls n \rs = \cos\lp \frac{2\pi n}{N} \rp$. \vspace{3cm}
        \item $x\ls n \rs = \cos\lp \Omega_1 n \rp$ and $y\ls n \rs = \cos\lp \Omega_2 n \rp$. \vspace{3cm}
        \item $x\ls n \rs = \cos\lp \frac{2\pi k_1 n}{N} \rp$ and $y\ls n \rs = \cos\lp \frac{2\pi k_2 n}{N} \rp$. \vspace{3cm}
    \end{enumerate}
    \newpage

    \item Consider the signals $x_1[n] = \frac{1}{\sqrt{2}}\bmx 1 & 1 & 0\emx^\top$,  $x_2[n] = \frac{1}{\sqrt{2}}\bmx -1 & 1 & 0\emx^\top$, and $x_3[n] = \bmx 0 & 0 & 1\emx^\top$.

    Find the representation of the signal $w[n] = \bmx 2 & 3 & -4 \emx^\top$ in terms of $x_1[n]$, $x_2[n]$, and $x_3[n]$.

    \newpage

    \item Consider a signal $x\lp t \rp = \sin\lp 12\pi t + 0.25\pi\rp$. Find the digital frequency of the signal when $x\lp t\ rp$ is sampled at the following sampling frequencies.
    \begin{enumerate}
        \item $F_s = 50Hz$
        \item $F_s = 18Hz$
        \item $F_s = 1Hz$
    \end{enumerate}
\end{enumerate}


\end{document}