% Copyright 2004 by Till Tantau <tantau@users.sourceforge.net>.
%
% In principle, this file can be redistributed and/or modified under
% the terms of the GNU Public License, version 2.
%
% However, this file is supposed to be a template to be modified
% for your own needs. For this reason, if you use this file as a
% template and not specifically distribute it as part of a another
% package/program, I grant the extra permission to freely copy and
% modify this file as you see fit and even to delete this copyright
% notice. 

\documentclass[aspectratio=169]{beamer}
%\documentclass{beamer}

\setbeamersize{text margin left=10mm, text margin right=10mm}

\defbeamertemplate{headline}{my header}{%
\vskip1pt%
\makebox[0pt][l]{\,\insertshortauthor}%
\hspace*{\fill}\insertshorttitle/\insertshortsubtitle\hspace*{\fill}%
\llap{\insertpagenumber/\insertpresentationendpage\,}
}
\setbeamertemplate{headline}[my header]

\usepackage{graphicx}
\usepackage{soul}
\usepackage{tkz-euclide}
\usetikzlibrary{calc}
\usepackage[]{algorithm2e}
\usepackage{changepage}
\usepackage{amssymb}
\usepackage{xcolor}
\usepackage{mathtools}
\usepackage{tcolorbox}
\usepackage{tikz}
\usetikzlibrary{arrows}
\usepackage{textcomp}
\usetikzlibrary{shapes}
\usepackage{tikz-3dplot}
\usepackage{tkz-euclide}
\usepackage{circuitikz}
\usepackage{pgfplots}
\pgfplotsset{width=7cm,compat=1.8}
\usetikzlibrary{positioning}
\usepackage[tikz]{bclogo}
\presetkeys{bclogo}{
ombre=false,
epBord=0,
couleur = black!10!white,
couleurBord = red,
arrondi = 0.1,
logo=,
}{}
% \usepackage[math]{cellspace}
% \cellspacetoplimit 4pt
% \cellspacebottomlimit 4pt
%\usetikzlibrary{arrows.meta}
% sqare of half axes
\newcommand{\asa}{3}
\newcommand{\bsa}{1}
\newcommand{\csa}{0.25}
% view angle
\tdplotsetmaincoords{70}{135}

%\setbeamertemplate{itemize items}{-}

%\usepackage{helvet}
\usefonttheme{professionalfonts} % using non standard fonts for beamer
%\usefonttheme{serif} % default family is serif
%\usepackage{fontspec}
%\setmainfont{Liberation Serif}

% There are many different themes available for Beamer. A comprehensive
% list with examples is given here:
% http://deic.uab.es/~iblanes/beamer_gallery/index_by_theme.html
% You can uncomment the themes below if you would like to use a different
% one:
%\usetheme{AnnArbor}
%\usetheme{Antibes}
%\usetheme{Bergen}
%\usetheme{Berkeley}
%\usetheme{Berlin}
%\usetheme{Boadilla}
%\usetheme{boxes}
%\usetheme{CambridgeUS}
%\usetheme{Copenhagen}
%\usetheme{Darmstadt}
%\usetheme{default}
%\usetheme{Frankfurt}
%\usetheme{Goettingen}
%\usetheme{Hannover}
%\usetheme{Ilmenau}
%\usetheme{JuanLesPins}
%\usetheme{Luebeck}
%\usetheme{Madrid}
%\usetheme{Malmoe}
%\usetheme{Marburg}
%\usetheme{Montpellier}
%\usetheme{PaloAlto}
%\usetheme{Pittsburgh}
%\usetheme{Rochester}
%\usetheme{Singapore}
%\usetheme{Szeged}
%\usetheme{Warsaw}

% Definition of blocks:
\tikzset{%
  block/.style    = {draw, thick, rectangle, minimum height = 3em,
    minimum width = 3em},
  sum/.style      = {draw, circle, node distance = 2cm}, % Adder
  input/.style    = {coordinate}, % Input
  output/.style   = {coordinate} % Output
}
% Defining string as labels of certain blocks.
\newcommand{\suma}{\Large$+$}
\newcommand{\inte}{$\displaystyle \int$}
\newcommand{\derv}{\huge$\frac{d}{dt}$}

\def\mf{\ensuremath\mathbf}
\def\mb{\ensuremath\mathbb}
\def\lp{\ensuremath\left(}
\def\rp{\ensuremath\right)}
\def\lv{\ensuremath\left\lvert}
\def\rv{\ensuremath\right\rvert}
\def\lV{\ensuremath\left\lVert}
\def\rV{\ensuremath\right\rVert}
\def\lc{\ensuremath\left\{}
\def\rc{\ensuremath\right\}}
\def\ls{\ensuremath\left[}
\def\rs{\ensuremath\right]}
\def\bmx{\ensuremath\begin{bmatrix*}[r]}
\def\emx{\ensuremath\end{bmatrix*}}
\def\bmxc{\ensuremath\begin{bmatrix*}[c]}
\def\t{\lp t\rp}
\def\k{\ls k\rs}


\newcommand{\demoex}[2]{\onslide<#1->\begin{color}{black!60} #2 \end{color}}
\newcommand{\demoexc}[3]{\onslide<#1->\begin{color}{#2} #3 \end{color}}
\newcommand{\anim}[3]{\onslide<#1->{\begin{color}{#2!60} #3 \end{color}}}
\newcommand{\ct}[1]{\lp #1\rp}
\newcommand{\dt}[1]{\ls #1\rs}
\newcommand{\cols}[2]{\begin{columns}[#1] #2 \end{columns}}
\newcommand{\col}[2]{\begin{column}{#1} #2 \end{column}}


% \newenvironment{rcases}
% {\left.\begin{aligned}}
% {\end{aligned}\right\rbrace}


\title{Linear Systems}

% A subtitle is optional and this may be deleted
\subtitle{Stability}

\author{Sivakumar Balasubramanian}
% - Give the names in the same order as the appear in the paper.
% - Use the \inst{?} command only if the authors have different
%   affiliation.

\institute[Christian Medical College] % (optional, but mostly needed)
{
  \inst{}%
  Department of Bioengineering\\
  Christian Medical College, Bagayam\\
  Vellore 632002
}
% - Use the \inst command only if there are several affiliations.
% - Keep it simple, no one is interested in your street address.

\date{}
% - Either use conference name or its abbreviation.
% - Not really informative to the audience, more for people (including
%   yourself) who are reading the slides online

\subject{Lecture notes on linear systems}
% This is only inserted into the PDF information catalog. Can be left
% out. 

% If you have a file called "university-logo-filename.xxx", where xxx
% is a graphic format that can be processed by latex or pdflatex,
% resp., then you can add a logo as follows:

% \pgfdeclareimage[height=0.5cm]{university-logo}{university-logo-filename}
% \logo{\pgfuseimage{university-logo}}

% Delete this, if you do not want the table of contents to pop up at
% the beginning of each subsection:
\AtBeginSubsection[]
{
  \begin{frame}<beamer>{Outline}
    \tableofcontents[currentsection,currentsubsection]
  \end{frame}
}

% Let's get started
\begin{document}

\pgfplotsset{
  compat=1.8,
  colormap={whitered}{color(0cm)=(white); color(1cm)=(orange!75!red)}
}

\begin{frame}
  \titlepage
\end{frame}


\begin{frame}[t]{Internal stability}
\begin{itemize}
    \item There are two types of stability one can associate with a system $\dot{\mf{x}}\ct{t} = \mf{f}\ct{\mf{x}\ct{t}, \mf{u}\ct{t}}$ -- \textbf{Internal stability} and \textbf{Input-Output stability}.

    \item \textbf{Internal stability}: Deals with the stability of the zero-input response of the system states, i.e. $\dot{\mf{x}}\ct{t} = \mf{f}\ct{\mf{x}\ct{t}}$.

    \item An \textit{equilibrium point} $\mf{x}_e$ of this system is defined as a point in the state space where, $\dot{\mf{x}}\ct{t} = \mf{f}\ct{\mf{x}_e} = \mf{0}$, i.e. if the system starts in this state, it stays in that state for all time.

    \item In the case of linear systems, we have $\mf{A}\mf{x}_e = \mf{0}$. The nullspace of $\mf{A}$ is the set of all equilibrium points of the linear system.
\end{itemize}
\end{frame}


\begin{frame}[t]{Internal stability}
Find the equilibrium points for the following systems with $\mf{f}\ct{\mf{x}\ct{t}}$: (a) $\bmxc x_2 \\ \sin x_1\emx$; (b) $\bmx x_1 + 2x_2\\ 2x_1 + 3x_2\emx$; (c) $\bmx -x_1 + x_2\\ x_1 - x_2\emx$; and (d) $\bmx 0 \\ 0\emx$.
\end{frame}


\begin{frame}[t]{Internal stability}
\begin{itemize}
    \item Definition of stability in the Lyapunov sense for linear systems: 
    \begin{itemize}
      \item The zero-input response of a linear system $\dot{\mf{x}}\ct{t} = \mf{A}\mf{x}\ct{t}$ is \textit{stable or marginally stable} if every finite initial condition $\mf{x}\ct{0^-}$ results in a bounded state trajectory $\mf{x}\ct{t}$ $\forall t \geq 0$.
      \[ \lV \mf{x}\ct{t}\rV \leq d, \,\,\, \forall t \geq 0 \]

      \item The zero-input response is \textit{asymtotically stable} if everyf initial condition $\mf{x}\ct{0^-}$ results in a bounded state trajectory $\mf{x}\ct{t}$ that coverges to $0$ as $t \to \infty$.
      \[ \lV \mf{x}\ct{t}\rV \leq d \text{ and } \lim_{t \to \infty} \lV \mf{x}\ct{t}\rV = 0 \]
    \end{itemize}
\end{itemize}
\end{frame}


\begin{frame}[t]{Internal stability}
\begin{itemize}
    \item The system $\dot{\mf{x}}\ct{t} = \mf{A}\mf{x}\ct{t}$ is marginally stable if and only if all eigenvales of $\mf{A}$ have either zero or negative real parts, and the eigenvalues with zero real parts have the same algebraic and geometric multiplicity.

    \item  The system $\dot{\mf{x}}\ct{t} = \mf{A}\mf{x}\ct{t}$ is asymptotically stable if and only if all eigenvales of $\mf{A}$ have  negative real parts.
\end{itemize}
\end{frame}


\begin{frame}[t]{Internal stability}
\begin{itemize}
    \item Consider the solution, $\mf{x}\ct{t} = e^{t\mf{A}}\mf{x}\ct{0^-}, \,\,\, t \geq 0$, and $\mf{A} = \mf{V}\mf{J}\mf{V}^{-1}$.
    \[ \lV \mf{x}\ct{t} \rV = \lV e^{t\mf{A}}\mf{x}\ct{0^-} \rV \leq \lV e^{t\mf{J}} \rV \lV \mf{x}\ct{0^-} \rV\]

    \item When $\mf{A}$ is diagonalizable ($\lambda_i$ are the eigenvalues of $\mf{A}$),
    \begin{itemize}
        \item $ \lV \mf{x}\ct{t} \rV \leq  e^{\sigma t} \lV \mf{x}\ct{0^-} \rV$, where $\sigma = \max_i \Re \lc \lambda_i \rc$.
        \item When $\sigma = 0$, $\lV \mf{x}\ct{t}\rV$ is bounded $\forall t \geq 0$.
        \item When $\sigma < 0$, $\lim_{t \to \infty} \lV \mf{x}\ct{t}\rV = 0$.
    \end{itemize}\vspace{0.2cm}
\end{itemize}
\end{frame}


\begin{frame}[t]{Internal stability}
\begin{itemize}
    \item When $\mf{A}$ is not diagonalizable, then $\mf{J}$ is block diagonal.
    \begin{itemize}
        \item Consider the $i^{th}$ Jordan block, $\mf{J}_i = \lambda_i \mf{I} + \mf{N}$, Thus, $e^{t\mf{J}_i} = e^{\lambda_i t\mf{I}}e^{t\mf{N}} \implies $ $ \lV \mf{x}\ct{t} \rV \leq e^{\sigma_i t} \lV e^{t\mf{N}} \rV \lV \mf{x}\ct{0^-} \rV$

        \item When $\sigma_i = 0$, $\lV e^{t\mf{N}} \rV$ grows with time, and thus $\mf{x}\ct{t}$ is not bounded. 

        \item When $\sigma_i < 0$, the $e^{\sigma_i t}$ term does not allow $\mf{x}\ct{t}$ to grow. 
    \end{itemize}
\end{itemize}
\end{frame}
 

\begin{frame}[t]{Internal stability}
Comment of the stability: (a) $\bmx -1 & 1\\ 0 & -2\emx$; (b) $\bmx 0 & -1\\ 1 & 0\emx$; (c) $\bmx 1 & 2\\ 0 & -2\emx$;  and (d) $\bmx -1 & -1 & 1\\ 0 & -2 & 1\\ 0 & 0 & 0\emx$
\end{frame}
 

\begin{frame}[t]{Internal stability -- Lyapunov stability criteria}
\begin{itemize}
    \item A general approach to evaluating the the stability of a dynamic system $\dot{\mf{x}}\ct{t} = \mf{f}\ct{\mf{x}\ct{t}}$ was proposed by Lyapunov. 

    \item Stability is inferred by looking at the energy associated with a system, and how it changes as the system evolves. i.e, whether the system dissipates, conserves or generates energy with time.

    \item The idea of the energy associated with the system and its change with time is captured through a \textit{Lyapunov function} $V\ct{\mf{x}}: \mb{R}^n \to \mb{R}_{\geq 0}$,
    \[ V\ct{\mf{0}} = 0 \,\, \text{ and } V\ct{\mf{x}} > 0 \,\, \forall \mf{x} \neq \mf{0}, \,\,\, \text{ and } \dot{V}\ct{\mf{x}} \leq 0 \]
\end{itemize}
\end{frame}


\begin{frame}[t]{Internal stability -- Lyapunov stability criteria}
\begin{itemize}
    \item $\dot{V}\ct{\mf{x}} = \ct{\frac{\partial}{\partial\mf{x}}V\ct{\mf{x}}}\dot{\mf{x}} = \ct{\frac{\partial}{\partial\mf{x}}V\ct{\mf{x}}}\mf{f}\ct{\mf{x}}$ is the time rate of change of energy of the system. 

    \begin{itemize}
        \item Stable (marginally) systems conserve energy, i.e. $\dot{V}\ct{\mf{x}} = 0$.
        \item Asymptotically stable systems dissipate energy, i.e. $\dot{V}\ct{\mf{x}} < 0$.
        \item Unstable systems generate energy, i.e. $\dot{V}\ct{\mf{x}} > 0$.
    \end{itemize}

    \item For a given system, if we can find a Lyapunov function, then the system is stable or asymptotically stable if $\dot{V}\ct{\mf{x}} < 0$.
\end{itemize}
\end{frame}


\begin{frame}[t]{Internal stability -- Lyapunov stability criteria}
\begin{itemize}
    \item Consider, $\dot{\mf{x}}\ct{t} = \bmx 0 & 1 \\ -\frac{k}{m} & -\frac{b}{m} \emx\mf{x}\ct{t}$. The energy associated with this system is $V\ct{\mf{x}} = \frac{1}{2}kx_1^2 + \frac{1}{2}mx_2^2$ $\implies \dot{V}\ct{\mf{x}} = -b x_2^2$. Is this system stable?
\end{itemize}
\end{frame}


\begin{frame}{Internal stability -- Lyapunov stability criteria}
\begin{itemize}
    \item Consider a general LTI system, $\dot{\mf{x}}\ct{t} = \mf{A}\mf{x}\ct{t}$, with non-singular $\mf{A}$.

    A necessary and sufficient condition for this system to be asymptotically stable is for a given symmetric, positive definite matrix $\mf{Q}$, there exists a symmetric, positive definite matrix $\mf{P}$ such that
    \[ \mf{A}^T\mf{P} + \mf{P}\mf{A} = - \mf{Q} \] 

    \item We can arbitrarily choose $\mf{Q}$ and solve for $\mf{P}$. The positive definiteness of $\mf{P}$ is a necessary and sufficient condition for the asymptotic stability of the LTI system.
\end{itemize}
\end{frame}


\begin{frame}[t]{Internal stability -- Lyapunov stability criteria}
Is this system asymptotically stable? $\dot{\mf{x}}\ct{t} = \bmx 1 & 2 \\ -1 & 2 \emx\mf{x}\ct{t}$
\end{frame}


\begin{frame}[t]{Internal stability -- Discrete-time LTI systems}
\begin{itemize}
    \item The system $\mf{x}\dt{k+1} = \mf{A}\mf{x}\dt{k}$ is marginally stable if and only if all eigenvales of $\mf{A}$ either of magnitude 1 or less than 1, and the eigenvalues with magnitude 1 have the same algebraic and geometric multiplicity.

    \item  The system $\mf{x}\dt{k+1} = \mf{A}\mf{x}\dt{k}$ is asymptotically stable if and only if all eigenvales of $\mf{A}$ have magnitude less than 1.

    \item $\mf{x}\dt{k} = \mf{A}^k\mf{x}\dt{0}, \, k > 0$, and $\mf{A} = \mf{V}\mf{J}\mf{V}^{-1}$
    \[ \lV \mf{x}\dt{k} \rV = \lV \mf{A}^k\mf{x}\ct{0^-} \rV \leq \lV \mf{J}^k \rV \lV \mf{x}\ct{0^-} \rV\]
\end{itemize}
\end{frame}


\begin{frame}[t]{Internal stability -- Discrete-time LTI systems}
When $\mf{A}$ is diagonalizable ($\lambda_i$ are the eigenvalues of $\mf{A}$),
\begin{itemize}
    \item $\lV \mf{x}\dt{k} \rV \leq  \lv \lambda \rv^k \lV \mf{x}\dt{0} \rV$, where $\lambda = \max_i \lv \lambda_i \rv$.
    \item When $\lv \lambda \rv = 1$, $\lV \mf{x}\dt{k}\rV$ is bounded $\forall k > 0$.
    \item When $\lv \lambda \rv < 1$, $\lim_{k \to \infty} \lV \mf{x}\dt{k}\rV = 0$.
\end{itemize}
\end{frame}


\begin{frame}[t]{Internal stability -- Discrete-time LTI systems}
When $\mf{A}$ is not diagonalizable, then $\mf{J}$ is block diagonal.
\begin{itemize}
    \item Consider the $i^{th}$ Jordan block, $\mf{J}_i^k = \ct{\lambda_i \mf{I} + \mf{N}}^k = \sum_{l=0}^k \frac{k!}{(k - l)!l!}\lambda_i^l\mf{N}^{k - l}$

    \item When $\lv \lambda_i\rv = 1$, $\lV \mf{J}_i^k \rV$ grows with time, and thus $\mf{x}\dt{k}$ is not bounded.

    \item When $\lv \lambda_i\rv < 1$, the $\lambda_i^l$ term does not allow $\mf{x}\dt{k}$ to grow.
\end{itemize}
\end{frame}


\begin{frame}[t]{Internal stability -- Lyapunov stability criteria (discrete-time system)}
\begin{itemize}
    \item For a discrete-time system, $\mf{x}\dt{k + 1} = \mf{A}\mf{x}\dt{k}$, we again start with a scalar, positive definite, continuous (``energy'' like) function $V\ct{\mf{x}}$.

    \item The rate of change of energy is captured by successive differences in the values of $V\ct{\mf{x}}$ for different values of $k$, i.e. $\Delta V\ct{\mf{x}} = V\ct{\mf{x}\dt{k + 1}} - V\ct{\mf{x}\dt{k}}$.
    \begin{itemize}
      \item Stable (marginally) systems conserve energy, i.e. $\Delta V\ct{\mf{x}} = 0$.
      \item Asymptotically stable systems dissipate energy, i.e. $\Delta V\ct{\mf{x}} < 0$.
      \item Unstable systems generate energy, i.e. $\Delta V\dt{\mf{x}} > 0$.
    \end{itemize}
\end{itemize}
\end{frame}


\begin{frame}{Internal stability -- Lyapunov stability criteria (discrete-time system)}
\begin{itemize}
    \item A necessary and sufficient condition for this system $\mf{x}\dt{k+!} = \mf{A}\mf{x}\dt{k}$ to be asymptotically stable is for a given symmetric, positive definite matrix $\mf{Q}$, there exists a symmetric, positive definite matrix $\mf{P}$ such that
    \[ \mf{A}^T\mf{P}\mf{A} + \mf{P} = - \mf{Q} \] 

    \item We can arbitrarily choose $\mf{Q}$ and solve for $\mf{P}$. The positive definiteness of $\mf{P}$ is a necessary and sufficient condition for the asymptotic stability of the LTI system.
\end{itemize}
\end{frame}


\begin{frame}[t]{Internal stability -- Lyapunov stability criteria (discrete-time system)}
Is this system asymptotically stable? $\mf{x}\dt{k+1} = \bmx 1 & 2 \\ -1 & 2 \emx\mf{x}\dt{k}$
\end{frame}


\begin{frame}[t]{Input-Output stability}
\begin{itemize}
    \item Input-output stability or external stability deals with the forced response of a system, assuming the system is relaxed.

    \item Input-output stability is also known as BIBO (bounded input, bounded output) stability, i.e. a bounded input $\mf{u}\ct{t}$ applied to the system produces a bounded output $\mf{y}\ct{t}$.
\end{itemize}
\end{frame}


\begin{frame}[t]{Input-Output stability}
\begin{itemize}
    \item A single input, single output (SISO) LTI system with impulse response $h\ct{t}$ is BIBO stable, if and only if
    \[ \int_{0}^{\infty} \lv h\ct{t}\rv dt < \infty \]

    When $h\ct{t}$ is not absolutely integrable, then we are not guaranteed that bounded inputs will produce bounded outputs.

    \item A SISO system with a rational transfer function $H\ct{s}$ is BIBO stable if and only if all its poles lie in the left half of the $s$-plane.
    \[ H\ct{s} = \frac{B\ct{s}}{A\ct{s}} \xrightarrow{\mathcal{L}^{-1}} h\ct{t} \text{ contains } e^{p_i t}, te^{p_i t}, \ldots t^{m-1}e^{p_i t}\]
\end{itemize}
\end{frame}


\begin{frame}[t]{Input-Output stability}
\begin{itemize}
    \item In the case of a muti-input, multi-output (MIMO) LTI system, the impulse response and transfer function matrices are given by,
    \[ \mf{G}\ct{t} = \mf{C}e^{t\mf{A}}\mf{B} + \mf{D}\delta\ct{t} \text{ and } \mf{H}\ct{s} = \mf{C}\ct{s\mf{I} - \mf{A}}^{-1}\mf{B} + \mf{D} \]

    \item A MIMO system is BIBO stable, if and only if each element of the impulse response matrix $\mf{G}\ct{t}$ is absolutely integrable.
    \[ \int_{0}^\infty \lv g_{ij}\ct{t}\rv dt < \infty , \,\,\, \forall 1 \leq i, j \leq n \]
\end{itemize}
\end{frame} 


\begin{frame}[t]{Input-Output stability}
\begin{itemize}
    \item A MIMO LTI system is BIBIO stable, if and only if the poles of each element of the transfer function matrix $H\ct{s}$ lie in the left half of the $s$-plane.

    Even if we have eigenvalue that have positive real parts, the system migth still be BIBO stable because of pole-zero cancellations in the individual elements of $\mf{G}\ct{s}$.
\end{itemize}

Is this system externally stable? $\mf{A} = \bmx 1 & 0\\ 1 & -1\emx$, $\mf{B} = \bmx 1 \\ 0\emx$, $\mf{C} = \bmx 1 & -2\emx$. Is this system internally stable?
\end{frame} 


\begin{frame}{Input-Output stability (discrete-time system)}
\begin{itemize}
    \item A SISO discrete-time LTI system with impulse response $h\dt{k}$ is BIBO stable, if and only if
    \[ \sum_{k=0}^{\infty} \lv h\dt{k}\rv < \infty \]

    \item A SISO system with a rational transfer function $H\ct{z}$ is BIBO stable if and only if all its poles lie within the unit circle $\lv z\rv = 1$.
    \[ H\ct{z} = \frac{B\ct{z}}{A\ct{z}} \xrightarrow{\mathcal{L}^{-1}} h\dt{k} \text{ contains } p_i^k, kp_i^k, \ldots k^{m-1}p_i^k \]
\end{itemize}
\end{frame} 


\begin{frame}{Input-Output stability (discrete-time system)}
\begin{itemize}
    \item A MIMO discret-time LTI system is BIBO stable, if and only if each element of the impulse response matrix $\mf{G}\dt{k}$ is absolutely summable.
    \[ \sum_{k=0}^\infty \lv g_{ij}\dt{k}\rv < \infty , \,\,\, \forall 1 \leq i, j \leq n \]

    \item A MIMO discrete-time LTI system is BIBO stable, if and only if the poles of each element of the transfer function matrix $H\ct{z}$ lie in the unit circle.
\end{itemize}
\end{frame} 
\end{document}